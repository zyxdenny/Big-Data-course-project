\documentclass{beamer}
\usetheme{CambridgeUS}
\usecolortheme{dolphin}

\usepackage[greek,english]{babel}
\usepackage[OT2,OT1]{fontenc}
\usepackage[utf8]{inputenc}
\usepackage{hyperref}
\usepackage{geometry}
\usepackage{enumerate}
\usepackage{multicol}
\usepackage{multirow}
\usepackage{array}
\usepackage{ulem}
\usepackage{graphicx}
\usepackage{subfigure}
\usepackage{algorithm}
\usepackage{algorithmicx}
\usepackage{algpseudocode}
\renewcommand{\algorithmicrequire}{\textbf{Input:}}
\renewcommand{\algorithmicensure}{\textbf{Output:}}
\usepackage{amsmath}
\usepackage{amssymb}
\usepackage{mathrsfs}
\usepackage{latexsym}
%\usepackage[scheme=plain]{ctex}
%\usepackage{tikz}
\usepackage{verbatim}

\newenvironment{command}{\begin{block}{Command}}{\end{block}}
\newcommand{\samplebegin}[1]{\structure{\textbackslash begin}\{#1\}}
\newcommand{\sampleend}[1]{\structure{\textbackslash end}\{#1\}}
\newcommand{\samplecommand}[1]{\alert{\textbackslash #1}}
\newcommand{\samplecolorbox}[1]{\fcolorbox{black}{#1}{\color{#1}{\tiny{\phantom{0000}}}} \small{#1}}
\newcommand{\sampletext}[2]{\samplecommand{#1} - {#2{Sample Text}}}
\newcommand{\sampleaccent}[3]{\samplecommand{#1}\{#3\}\quad #2{#3}}
\newcommand{\samplesymbol}[2]{\samplecommand{#1}\quad #2}

% Copied from lshort.sty
%
% Symbol Entry for Math Symbol Tables
%
\newcommand{\X}[1]{$#1$&\texttt{\string#1}\hspace*{1ex}}
% normal text ....
\newcommand{\SC}[1]{#1&\texttt{\string#1}\hspace*{1ex}}
% for accents in text mode
\newcommand{\A}[1]{#1&\texttt{\string#1}\hspace*{1ex}}
\newcommand{\B}[2]{#1#2&\texttt{\string#1{} #2}\hspace*{1ex}}

\newcommand{\W}[2]{$#1{#2}$&
  \texttt{\string#1}\texttt{\string{\string#2\string}}\hspace*{1ex}}
\newcommand{\Y}[1]{$\big#1$ &\texttt{\string#1}}  %
% Mathsymbol Table
\newsavebox{\symbbox}
\newenvironment{symbols}[1]%
{\par\vspace*{2ex}
\renewcommand{\arraystretch}{1.1}
\begin{lrbox}{\symbbox}
\hspace*{4ex}\begin{tabular}{@{}#1@{}}}%
{\end{tabular}\end{lrbox}\makebox[\linewidth]{\usebox{\symbbox}}\par\medskip}
%
% Some commands for helping with INDEX creation
%
\newcommand{\bs}{\symbol{'134}}%Print backslash
%\newcommand{\bs}{\ensuremath{\mathtt{\backslash}}}%Print backslash
% Index entry for a command (\cih for hidden command index
\newcommand{\eei}[1]{%
\index{extension!\texttt{#1}}\texttt{#1}}
% probably add handling of period like handling of \ in \ci
\newcommand{\fni}[1]{%
\index{font!#1@\texttt{\bs#1}}%
\index{#1@\texttt{\hspace*{-1.2ex}\bs #1}}\texttt{\bs #1}}
\newcommand{\cih}[1]{%
\index{commands!#1@\texttt{\bs#1}}%
\index{#1@\texttt{\hspace*{-1.2ex}\bs #1}}}
\newcommand{\ci}[1]{\cih{#1}\texttt{\bs #1}}
%Package
\newcommand{\paih}[1]{%
\index{packages!#1@\textsf{#1}}%
\index{#1@\textsf{#1}}}
\newcommand{\pai}[1]{%
\paih{#1}\textsf{#1}}
% Index entry for an environment
\newcommand{\ei}[1]{%
\index{environments!\texttt{#1}}%
\index{#1@\texttt{#1}}%
\texttt{#1}}
% Indexentry for a word (Word inserted into the text)
\newcommand{\wi}[1]{\index{#1}#1}

\usepackage{minted}
\usemintedstyle{autumn}
\setminted{
    linenos,
    breaklines,
    fontsize=\footnotesize,
    tabsize=4,
    xleftmargin=2em
}
\setmintedinline{
	fontsize=\footnotesize
}
\newcommand{\LC}[1]{\mintinline{latex}{ #1 }}
\newcommand{\LCL}[1]{\begin{minted}{latex}{
#1
}
\end{minted}
}

\newcommand{\myhref}[1]{{\color{blue}\url{#1}}}
\renewcommand{\multirowsetup}{\centering}  

\title{ECE4721J --- Methods and Tools for Big Data}
\author{p1team-01}
\date{\today}
\subtitle{Million Song Dataset (MSD) Analysis Tools}
\institute{UM-JI (Summer 2022)}

\begin{document}

\begin{frame}
	\titlepage
\end{frame}

\section{HDF5 file process}
\begin{frame}
	\tableofcontents[currentsection,hideothersubsections]
\end{frame}

\subsection{Basic Setup}
\begin{frame}
	\frametitle{Overview}
	\begin{itemize}
		\item Maven-managed Java Project
		\begin{itemize}
		\item Cross-platform 
		\item Easy to install new packages (compared with c++) 
		\item Easy to manage different packages (compared with Python \mintinline{Python}{import})
		\end{itemize}
		\item Avro
		\begin{itemize}
		\item Easy to be integrate into Java project as part of Apache Ecosystem
		\item Can be easy accessd and processed by Drill and Spark easily
		\item Compact small files together to avoid waste of memory in HDFS
		\item More freedom in data retrieve
		\end{itemize}
	
	\end{itemize}	
\end{frame}

\subsection{HDF5 file related feature}
\begin{frame}
	\frametitle{HDF5 file related feature}
	HDF5 file related functions are implemented within \mintinline{Java}{H5_parser} class
	\begin{itemize}
		\item \mintinline{Java}{H5_parser.recursivePrintGroup}: Print all the group information stored in selected h5 file
		\item \mintinline{Java}{H5_parser.printData}: Print all the data with its paths. The compound data will be print separately.
		\item \mintinline{Java}{H5_parser.printDataType}: Print the data type of each field in the HDF5 file.
	\end{itemize}
	
\end{frame}

\subsection{Avro related feature}
\begin{frame}
	\frametitle{Overview}
	Basically, we provide three kinds of avro compact method.
	\begin{itemize}
		\item \mintinline{Java}{song}: Compact all the information with respect to its field and the final results will be separated into analysis, metadata and musicbrainz in Drill
		\item \mintinline{Java}{song_summary}: Compact only the information required for Drill process to provide Drill-friendly avro file
		\item \mintinline{Java}{artists}: Compact only the information required for constructing graph for advanced analysis feature. You can manually generate for a test but not recommended.
	\end{itemize}
	
\end{frame}
\begin{frame}
	\frametitle{Avro related feature}
	Avro related functions are implemented within \mintinline{Java}{CompactSmallFiles} class
	\begin{itemize}
		\item \mintinline{Java}{CompactSmallFiles.serialize}: compact all the h5 files within the folder in \mintinline{Java}{song} mode
		\item \mintinline{Java}{CompactSmallFiles.serializeSummary}: compact all the h5 files within the folder in \mintinline{Java}{song_summary} mode
		\item \mintinline{Java}{CompactSmallFiles.serializeArtists}: compact all the h5 files within the folder in \mintinline{Java}{artists} mode
		\item \mintinline{Java}{CompactSmallFiles.serializeArtists_N}: compact defined number of h5 files within the folder in \mintinline{Java}{artists} mode
		\item \mintinline{Java}{CompactSmallFiles.readDir}: store all the h5 file into the avro process class within the folder and print the number of files store
	\end{itemize}
	
\end{frame}


\section{Map Reduce}
\begin{frame}
	\tableofcontents[currentsection,hideothersubsections]
\end{frame}

\subsection{Define a template}

\begin{frame}[fragile]
	\frametitle{template}
	Template as a special class see Page 331:
	\begin{minted}{cpp}
#include <iostream>
using namespace std;
template<class TYPE>
class Complex {
public:
Complex(){ R = I = (TYPE)0; }
Complex(TYPE real, TYPE img) {R=real;I=img;}
void PrintComplex() {cout<<R<<'+'<<I<<"i\n";}
private:
TYPE R, I;
};
\end{minted}
	Basic Usage, insert the TYPE with proper datatype :
	\begin{minted}{matlab}
Complex<float> c1; complex<int> c2;
typedef Complex<double> dcplx; dcplx c3;
\end{minted}
If you turn to cppreference, TYPE is more often written as T.
\end{frame}




\section{Spark}
\begin{frame}
	\tableofcontents[currentsection,hideothersubsections]
\end{frame}

\subsection{Basic concepts}
\begin{frame}
	\frametitle{Standard Template Library}
\end{frame}

\section{Reference}
\begin{frame}
	\frametitle{Reference}
	\begin{thebibliography}{}
		\bibitem{ref1}[1] 1
	\end{thebibliography}
\end{frame}



\end{document}